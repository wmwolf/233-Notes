\documentclass[10pt]{article}

%%%%%%%%%%%%%%%%%%%%%%%%%%%%%%%%%%%%%%%%%%%%%
% Package Inclusion and Document Formatting %
%%%%%%%%%%%%%%%%%%%%%%%%%%%%%%%%%%%%%%%%%%%%%
\usepackage
{geometry,amsmath,amsthm,mathrsfs,amssymb,graphicx,bm,hyperref,url,pdfsync,
fancyhdr, wrapfig}
\pagestyle{fancy}
\numberwithin{equation}{section}
%%%%%%%%%%%%%%%%%%%
% Custom Commands %
%%%%%%%%%%%%%%%%%%%
\newcommand{\n}{\noindent}
\newcommand{\norm}[1]{\left\lvert#1\right\rvert}
\newcommand{\avg}[1]{\left\langle#1\right\rangle}
\newcommand{\abs}[1]{\left\vert#1\right\vert}
\newcommand{\figref}[1]{Figure \ref{#1}}

\newcommand{\HI}{H\,I\ }
\newcommand{\HII}{H\,II\ }
\newcommand{\Htwo}{H$_2$\ }
%%%%%%%%%%%%%%%%%%%%%%%%%%
% Title Page Information %
%%%%%%%%%%%%%%%%%%%%%%%%%%

\title{Notes for PHYS 233: Interstellar Medium}
\author{Bill Wolf}
\date{\today}

\begin{document}

\vfill\maketitle\vfill \newpage

\tableofcontents \newpage

\section{Introduction}
\emph{Monday, January 6, 2013}\\

\n This course is called ``Interstellar Medium'' (ISM), but really it is about
the more general topic of the physics of diffuse gas in the universe. By
diffuse, we mean \emph{really} diffuse, down to number densities of 1
cm$^{-3}$, much more diffuse than the best vacuums in terrestrial laboratories.
This includes the interstellar gas in galaxies as well as the intergalactic gas
in between the galaxies.

\subsection{Organization of the ISM in the Milky Way} % (fold)
\label{sub:a_first_look}
Interstellar gas is what forms the stars in galaxies that are the dominant
sources of energy and light in the universe. Thus, understanding the physics of
the ISM helps us understand (and predict) the visible appearance of galaxies.\\

\noindent The space between the stars is occupied by gas, dust, cosmic rays,
electromagnetic radiation (from stars, the CMB, and radition from interstellar
matter), neutrinos, dark matter particles (whatever they are), and magnetic and
gravitational fields. Do note that dust is distinct from gas, as dust grains
have a lengthscale (micron) four orders of magnitude greater than gas particles
(angstrom).\\

\noindent In the Milky Way, there is roughly $10^{11}\ M_\odot$ in various
components:
\begin{itemize}
  \item $5\times 10^{10}\ M_\odot$ of stars
  \item $5\times 10^{10}\ M_\odot$ of dark matter
  \item $7\times 10^{9}\ M_\odot$ of interstellar gas
\end{itemize}
So the Milky Way is a rather gas-poor galaxy. Other galaxies like the
Magellanic clouds have most of their baryons in the gas phase. The obscuration
in the central disk of the Milky Way is produced by the dust along the midplane
(see color plate in 1.2, 1.65, 2.2 $\mu$m images). While the stars do form in a
very thin disk (thin compared to its width), the starlight heats up dust and
expands the dust disk to higher altitudes from the midplane.\\

\n The magnetic fields in the Milky Way can be detected by looking for
synchrotron emission (caused by electrons spiraling around galactic field
lines). We do observe this emission (see additional plate) and believe the
magnetic fields are caused by a dynamo effect due to the circulating galactic
material.

\subsubsection{Gas Content} % (fold)
\label{ssub:gas_content}
There is about $5\times 10^9\ M_\odot$ of hydrogen in the Milky Way, in the
following states
\begin{itemize}
  \item $2.9\times 10^9\ M_\odot$ of \HI gas (neutral)
  \item $1.12\times 10^9\ M_\odot$ of \HII gas (ionized)
  \item $0.84\times 10^9\ M_\odot$ of \Htwo gas (molecular)
\end{itemize}
We can track the presence of interstellar gas by looking at C$^+$ emission in
the Milky Way, which tracks hydrogen presence very well, though the weighting
is different since the emission strength is proportional to the density
squared.\\

\n Finally, we track galactic gas via CO emission, which is very concentrated
to the disk, but also has strong filaments outside the central disk of the
galaxy.\\

\n The interestellar medium exhibits turbulence that is the reult of supernovae
and other violent hydrodynamic events.
% subsubsection gas_content (end)

\paragraph{Phases of Interstellar Gas} % (fold)
\label{ssub:phases_of_interstellar_gas}
We categorize the ISM into various phases according to the ionization/molecular
state of its hydrogen (descriptions taken from Draine 1.1):
  \begin{itemize}
    \item \textbf{Warm \HI}: Neutral hydrogen gas at temperatures around $T \approx 10^{3.7}\ \mathrm{K}$. Typically at densities of $n_{\mathrm{H}}\approx 0.6\ \mathrm{cm^{-3}}$. Occupies about 40\% of the galactic disk. Often referred to as the warm neutral medium (WNM)
    \item \textbf{Cool \HI}: Neutral hydrogen gas at temperatures around $T \approx 100$ K with higher densities of $n_{\mathrm{H}}\approx 30\ \mathrm{cm^{-3}}$. Occupies around 1\% of the local ISM, sometimes referred to as the cold neutral medium (CNM)
    \item \textbf{Diffuse H}$_{2}$: Like the CNM, but with larger densities and column densities, allowing molecular hydrogen (\Htwo) to be more abundant.
    \item \textbf{Dense H}$_{2}$: Gravitationally bound clouds with number densities exceeding $10^3\ \mathrm{cm^{-3}}$. Distinguished by their dark appearance (strong optical extinction) in their central regions. Hosts of star formation. Not ``dense'' by terrestrial standards.
    \item \textbf{Ionized \HII} at $10^4$ K: Gas consisting largely of photoionized hydrogen, likely from ultraviolet radiation from a nearby massive star. May be dense material from a cloud (\HII region) or diffuse intercloud material (diffuse \HII). Lifetimes typically on the order of that of a massive star (Myrs). Extended diffuse photoionized regions, sometimes called the warm ionized medium (WIM), contain \emph{much} more total mass than the more visible \HII regions. Planetary nebulae (PNe) are also in this class.
    \item \textbf{Coronal Gas} (Ionized \HII at $\log T > 5.5$): Very hot, ionized gas that has been shock-heated by blastwaves from supernova explosions. It is mostly collisionally ionized with some exotic ionization states present. Typically very low density, filling about half the volume of the galactic disk. Cools on Myr timescales. Often referred to as the hot ionized medium (HIM).
  \end{itemize}
% paragraph phases_of_interstellar_gas (end)
\paragraph{Elemental Composition} % (fold)
\label{par:elemental_composition}
The gas is largely made of hydrogen and helium from the early universe, but an
additional 1-2\% (by mass) is heavy elements ($Z>2$), or to astronomers,
``metals''. The metallicity is actually a declining function from the center of
the galaxy outwards since it is the result of stellar processing, and the
center of the galaxy is more evolved then its relatively young outer regions.
The metallicity near the solar system is about half of that at the galactic
center. We know the local composition of the ISM by measuring solar
photospheric abundances and the composition of meteorites (See table 1.4 in
Draine).
% paragraph elemental_composition (end)

\subsubsection{Energy Content} % (fold)
\label{ssub:energy_content}
Energy in the CMB has a strange equipartition where nearly all components are
near 1 eV cm$^{-3}$. Some of this is self-sustaining, but since the CMB energy
density increases as $(1+z)^4$, its equality is coincidental. The rough energy
densities by categories are given in Table~\ref{tab:1.1}. Since the galactic
magnetic fields are driven by bulk motions, it is unsurprising that the
turbulent (bulk) and magnetic energies are nearly equal. If the cosmic ray
energy density were significantly greater, the magnetic fields could not
confine them, so they would simply escape the galaxy. Similarly, if the energy
from starlight were significantly higher, the ISM would expand and likely
dampen star formation, causing the near equipartion that we observe.
\begin{table}
  \centering
  \begin{tabular}{l l}
    Energy Type & Density (eV $\mathrm{cm^{-3}}$)\\
    \hline
    Thermal & 0.49\\
    Bulk Kinetic & 0.22\\
    Cosmic Ray & 1.39\\
    Magnetic & 0.89\\
    CMB & 0.265\\
    Far infrared from dust & 0.31 \\
    Starlight & 0.54
  \end{tabular}
  \caption{Energy densities in the ISM.}
  \label{tab:1.1}
\end{table}
If any of these components grows much larger than the gravitational binding
energy, then hydrostatic equilibrium is disrupted, driving a wind. All galaxies
drive winds at some time.
% subsubsection energy_content (end)
\subsubsection{Galactic Endgame} % (fold)
\label{ssub:galactic_endgame}
All of the ISM constituents are present between galaxies, and the same physical
processes apply to studying the intergalactic medium (IGM). In our present
state of the field, we can't account for the relative overabundance of dark
matter in the vicinity of a galaxy. It is thought that the ``missing baryons''
are in the circum-galactic material (CGM). The only place where the mass is
``right'' is in massive clusters. What is unknown is what sort of lengthscale
determines how far out this CGM might extend. Interestingly, the material
observed in the CGM contains metals, so either galactic winds are responsible
for pushing material processed by stellar evolution out of the galaxy, or very
massive population III stars formed at $z\sim 20-40$ to form the first metals.
% subsubsection galactic_endgame (end)
% subsection a_first_look (end)
\subsection{Basics of Diffuse Gas} % (fold)
\label{sub:basics_of_diffuse_gas}
Perhaps the prettiest regions of diffuse gas are the \HII regions, like the
Orion Nebula, which is illuminated by four central stars (the trapezium). The
typical \HII region has a number density around $1\ \mathrm{cm^{-3}}$ and a
temperature of $10^4$\ K. A big difference between these \HII regions (and in
fact most instances of interstellar gas) and stellar interiors is that they are
\emph{not} in local thermodynamic equilibrium (LTE).\\

\n The NLTE-ness of diffuse gas iss due tot he much, much lower collision rate
than in planetary atmospheres or stellar interiors. In order to achieve LTE, a
gas must satisfy the following conditions:
\begin{itemize}
  \item Population of excited states given by Boltzmann equilibrium
  \item Particle energies distributed according to the Maxwell distribution
  \item Ionization balance given by Saha Equation
  \item Photon energies described by Planck Function
\end{itemize}
In diffuse gas, we may indeed be able to identify a kinetic temperature that
gives the average kinetic energy of the particles, but this temperature need
not be the same as the Boltzmann temperature, which gives the occupation of
energy states in the system. Likewise, the ionization temperature, Planck
temperature (of a blackbody), and other such derived temperatures need not
match such a diffuse system.
% subsection basics_of_diffuse_gas (end)
%%%%%%%%%%%%%%%%%%%%%%%%%%%%%%
% Wednesday, January 8, 2014 %
%%%%%%%%%%%%%%%%%%%%%%%%%%%%%%
\section{Collisionally Excited Lines} % (fold)
\label{sec:collisionally_excited_lines} % (fold)
\n\textit{Wednesday, January 8, 2014}\\

\begin{table}
  \centering
  \begin{tabular}{l l l}
  Environment & Number Density (cm$^{-3}$) & Collision Timescale (s)\\
  \hline
  Earth Atmosphere & $2\times 10^{13}$ & $\sim 10^{-9}$\\
  Molecular Cloud & $10^6$ & $10^8$ (several days)\\
  ISM & 1 & $10^{11}$ ($10^4$ yr)
  \end{tabular}
  \caption{Some characteristic collision timescales.}
  \label{tab:2.1}
\end{table}
\n A typical timescale for collisions between particles can be estimated via
\begin{equation}
  \label{eq:coll:1} \tau = \frac{1}{\sigma v n}
\end{equation}
That is, it's essentially the inverse of the expected collision rate. Here
$\sigma$ is the cross section for interaction, $v$ is the relative velocity of
particles, and $n$ is the number density of particles. Some typical timescale
for various environments are given in Table~\ref{tab:2.1}.\\

\n Also of importance is the volumetric rate of emission or photons due to
recombination, which will be proportional to the product of electron and ion
number densities, or $r_{\mathrm{emis}}\propto n_{\mathrm{ion}}n_e$. We define
the \textbf{Emission Measure} as
\begin{equation}
  \label{eq:coll:2} \mathrm{EM} = \int n_e^2 ds
\end{equation}
where the integral is over some path in space. The ``natural'' units for this
measure is then pc cm$^{-6}$.
\paragraph{Example: Emission Measure of a Nova Shell} % (fold)
\label{par:example_emission_measure_of_a_nova_shell}
  For a nova shell with mass $\Delta M = 10^{-4}\ M_\odot$ expanding with a
  velocity $v=10^3\ \mathrm{km\ s^{-1}}$, we get an approximate emission
  measure of
  \begin{equation}
    \mathrm{EM} = (3\times 10^{-4}\ \mathrm{pc})(10^7\ \mathrm{cm^{-3}}) \sim
    3\times 10^{10}\ \mathrm{pc\ cm^{-6}}
  \end{equation}
  So at early times nova shells are very bright. We could play this game with
  other objects like planetary nebulae (PNe), \HII Regions, Diffuse warm-ionized
  gas, or circumgalactic gas to get an idea of how bright these objects would
  appear.
% paragraph example_emission_measure_of_a_nova_shell (end)
\subsection{Two-Level Atom} % (fold)
\label{sub:two_level_atom}
  % TODO Add energy level diagram
  If we consider an atom that has only two states, 1 and 2, with degeneracies
  $g_1$ and $g_2$ with an energy gap $\epsilon_{12}$, the collisional cross
  section is given by
  \begin{equation}
    \label{eq:coll:3} \sigma_{12}(E) = \frac{h^2}{8\pi m_e
    E}\frac{\Omega_{12}(E)}{g_1}
  \end{equation}
  where $\Omega$ is the \textbf{collision strength}, which is a dimensionless
  quantity with values around order unity. The \textbf{Principle of Detailed
  Balance}, which we have not yet proven, tells us that
  $\Omega_{12}=\Omega_{21}$.\\
  
  \n Indeed, the principle of detailed balance tells us that the rate of upwards and downward collisional transitions are equal, or
  \begin{align}
    \label{eq:coll:4}R_{12} &= R_{21}\\
    \label{eq:coll:5}n_1 n_e \alpha_{12} &= n_2 n_e \alpha_{21}
  \end{align}
  where $\alpha$ is the collisional excitation coefficient for that reaction. To derive this principle, we turn to statistical mechanics, and the concept of a \textbf{partition function}, which is the normalization to a boltzmann distribution:
  \begin{equation}
    \label{eq:coll:6} Z(T) \equiv \sum_s e^{-E(s)/(kT)}
  \end{equation}
  % TODO fix this mess
  where the sum is over all microstates. In a diulte gas, the partition function is actually a product of the particion functions from the internal system and that of the translational system:
  \begin{equation}
    \label{eq:coll:7} Z(T) = Z_{\mathrm{trans}}(T) \times Z_{\mathrm{int}}(T)
  \end{equation}
  Then we can define the particion function per unit volume in the usual way
  \begin{align}
    \label{eq:coll:8} f(T) &= \frac{Z(T)}{V}\\
    \label{eq:coll:9} &= \left(\frac{2\pi M_x kT}{h^3}\right)^{3/2}\times Z_{\mathrm{int}}(T)
  \end{align}
  Where we've glossed over some details present in Draine (Chapter 3). Using statistical mechanics, we are more or less assuming we are in LTE, so the number densities of various states for the reaction
  \begin{equation}
    \label{eq:coll:10} A + B \rightleftharpoons C
  \end{equation}
  is given in the usual way:
  \begin{equation}
    \label{eq:coll:11} \frac{n_{\mathrm{LTE}}(C)}{n_{\mathrm{LTE}}(A)
    n_{\mathrm{LTE}}(B)} = \frac{f(C)}{f(A)f(B)}
  \end{equation}
  which constrains the rate coefficients in a specific way. 
  % TODO something about how we don't need LTE
  Armed with this tool, we can actually try to compute collisional excitation rates:
  \begin{align}
    \label{eq:coll:12} R_{12} &= n_1 n_e \int_{E_{\mathrm{12}}}^\infty \sigma(v)v\,f(v)\,dv\\
    \label{eq:coll:13} R_{21} &= n_2 n_e \int_0^\infty \sigma(v)v\,f(v)\,dv
  \end{align}
  Now using the principle of detailed balance, we equate these rates and get 
  \begin{equation}
    \label{eq:coll:14} n_1e^{-E_{12}/(kT)}\frac{\Omega_{12}}{g_1} = \frac{n_2 \Omega_{21}}{g_2}
  \end{equation}
  Solving for the ratio of the number densities, we get
  \begin{equation}
    \label{eq:coll:15} \frac{n_2}{n_1} = \frac{\alpha_{12}}{\alpha_{21}} = \frac{\Omega_{21}}{g_1}\frac{g_2}{\Omega_{12}} e^{-E_{12}/(kT)}
  \end{equation}
  Or, more simply
  \begin{equation}
    \label{eq:coll:16} \frac{n_2}{n_1} = \frac{\Omega_{21}}{\Omega_{12}} \frac{g_2}{g_1} e^{-E_{12}/(kT)}
  \end{equation}
  Now if we are in LTE, we immediately recover $\Omega_{21} = \Omega_{12}$ since the Boltzmann ratio is already present. However, the collision strength is an intrinsic property of the ion, so this must be true in all cases. That is, we dont' require LTE to recover $\Omega_{12}=\Omega_{21}$ simply because we proved that it \emph{does} apply in LTE, and thus at all times.\\
  
\n In the low density limit, spontaneous decay dominates emission since collisionally excited transitions are few and far between. Then the transition rate can be written as
\begin{equation}
  \label{eq:einstein:1} n_en_1\alpha_{12} = n_2 A_{21}
\end{equation}
where $A$ is the \textbf{Einstein Coefficient} for spontaneous decay from state 2 to state 1. 
% TODO explain this next equation
\begin{equation}
  \label{eq:einstein:2} F_{12} = E_{12} A_{21} n_2
\end{equation}

Subbing in some Boltzmann algebra, $n_2$ can be expressed as
\begin{equation}
  \label{eq:einstein:3} n_2 = \frac{n_en_1}{A_{21}} \left( \frac{2\pi \hbar^4} {k m_e^3}\right)^{1/2} \frac{\Omega_{12}}{g_1} \frac{e^{-E_12}/(kT)}{\sqrt{T}}
\end{equation}
Plugging \eqref{eq:einstein:3} into \eqref{eq:einstein:2} and evaluating some of the constatns, we get
\begin{equation}
  \label{eq:einstein:4} F_{12} = E_{21} n_e n_1 \left(8.62942\times 10^{-6}\right) \frac{1}{\sqrt{T}} \frac{\Omega_{12}}{g_1}e^{-E_{12}/(kT)}
\end{equation}
Now in the high density limit, we do something similar, getting
\begin{equation}
  \label{eq:einstein:5} n_e n_1 \alpha_{12} = n_2 A_{21} + n_e n_2 \alpha{21}
\end{equation}
Now we get
\begin{equation}
  \label{eq:einstein:6} F_{12} = E_{21} A_{21} n_1 \frac{g_2}{g_1} e^{-E_{12} / (kT)}
\end{equation}
These two regimes beg for us to define a critical density that separates these two limits where $n_2A_{21} = n_2n_e \alpha_{21}$, which is, after some algebra,
\begin{equation}
  \label{eq:einstein:7} \boxed{n_{e, \mathrm{crit}} = \frac{A_{21}g_2 T^{1/2}} {\beta \Omega_{12}}}
\end{equation}
where 
\begin{equation}
  \label{eq:einstein:8} \beta \equiv \left[\frac{2\pi \hbar^4}{k m_e^3}\right]^{1/2}
\end{equation}
For forbidden lines, this critical density is in the range of $10^2-10^7\ \mathrm{cm^{-3}}$. For intercombination lines, the critical density is around $10^10\ \mathrm{cm^{-3}}$. For resonance lines (aka permitted lines), the critical density is $10^{15}\ \mathrm{cm^{-3}}$
% subsection two_level_atom (end)

%%%%%%%%%%%%%%%%%%%%%%%%%%%%
% Friday, January 10, 2014 %
%%%%%%%%%%%%%%%%%%%%%%%%%%%%
\subsection{Three-Level Atom} % (fold)
\label{sub:three_level_atom}
\textit{Friday, January 10, 2014}
\subsubsection*{Suggested References} % (fold)
\label{ssub:suggested_references}
\begin{itemize}
  \item Dopita \& Sutherland Chapter 3
\end{itemize}
% subsubsection suggested_references (end)
Now we consider the slightly more complicated three-level atom. Considering that the collisional excitation rate is, in general, given by
\begin{equation}
  \label{eq:three:1} C_{\mathrm{ij}} = n_e \alpha_{ij}
\end{equation}
for the appropriate rate factors $\alpha_{ij}$ for excitations from state $i$ to state $j$. Now we'll employ statistical equilibrium to figure out the equilibrium rates and concentrations. For level 3, we get the following
\begin{equation}
  \label{eq:three:2} n_1C_{13} + n_2 C_{23} = n_3(C_{31} + C_{32} + A_{32} + A_{31})
\end{equation}
now for level 2, we get
\begin{equation}
  \label{eq:three:3} n_1 C_{12} + n_3(C_{32} + A_{32}) = n_2(C_{21} + C_{23} + A_{21})
\end{equation}
and for normalization purposes, let's assume that the total number density is given by $n$:
\begin{equation}
  \label{eq:three:4} n_1 + n_2 + n_3 = n
\end{equation}
\eqref{eq:three:2} and \eqref{eq:three:3} essentially just equate the rates of reactions (collisional or spontaneous) for the ``formation'' of that energy state (left side) to the rates of ``destruction'' rates (right side).\\

\n Now we'll make some assumptions, namely that the energy levels are roughly equally spaced, or $E_{12} \approx E_{23}$. This tells us that the rate of excitations from state 1 to state 3 will be much lower than that from state 1 to state 2, or $C_{13}\gg C_{12}$.
% TODO Why do we make this assumption?
Now in a low density limit, we may ignore collisional excitation, which reduces our equations to a much cleaner form:
\begin{align}
  \label{eq:three:5} n_1 C_{13} &= n_3(A_{32}+A_{31})\\
  \label{eq:three:6} n_1 C_{12} + n_3A_{32} &= n_2A_{21}\\
  \label{eq:three:7} n_1 + n_2 + n_3 &= n
\end{align}
Solving these equations for $n_3$ and $n_2$ gives
\begin{align}
  \label{eq:three:9} n_3 &= \frac{n_1 C_{13}}{A_{32} + A_{31}}\\
  \label{eq:three:8} n_2 &= \left[n_1 C_{12} + \frac{A_{32}}{A_{32} + A_{31}} n_1 C_{13}\right]\frac{1}{A_{21}}\\
  \label{eq:three:10} &\approx \frac{n_1 C_{12}}{A_{21}}
\end{align}
Then the ratio of fluxes is 
\begin{equation}
  \label{eq:three:11} \frac{F_{32}}{F_{21}} = \frac{n_3 A_{32} E_{32}}{n_2 A_{21}E_{21}} = \cdots = \frac{C_{13}}{C_{12}} \frac{A_{32}}{A_{32} + A_{31}} \frac{E_{32}}{E_{21}}
\end{equation}
Now recalling that the $C$'s are related to our $\alpha$'s, via, for example,
\begin{equation}
  \label{eq:three:12} C_{13} = n_3 \left(\frac{2\pi \hbar^4}{k m_e^3}\right)^{1/2} \frac{1}{\sqrt{T}} \frac{\Omega_{13}}{g_1} e^{-{E_{13}}/kT},
\end{equation}
\eqref{eq:three:11} reduces to a function of a bunch of physical constants and the temperature
\begin{equation}
  \label{eq:three:13} \boxed{\frac{F_{32}}{F_{21}} = \frac{E_{32}}{E_{21}} \frac{A_{32}}{A_{32}+A_{21}} \frac{\Omega_{13}}{\Omega_{12}} e^{-E_{23}/kT}}
\end{equation}
So now given a line strength ratio, we can actually compute a temperature for that diffuse gas. Some examples of ions where these assumptions are valid are N$^+$, O$^{+2}$, Ne$^{+4}$, and S$^{+2}$.\\

\n Now instead if we assume that $E_{23} \ll E_{12}$ (i.e., the second and third states are very close in energy). This means, though it's not obvious, that $A_{32} \ll A_{31}$. Still in the low density limit (ignoring collisional de-excitation), we'll simplify \eqref{eq:three:2} and \eqref{eq:three:3} again:
\begin{align}
  \label{eq:three:14} n_1 C_{13} &= n_3(A_{32} + A_{31})\\
  \label{eq:three:15} & \approx n_3A_{31}\\
  \label{eq:three:16} n_1 C_{12} &= n_2 A_{21}
\end{align}
Then the flux ratios are simply
\begin{equation}
  \label{eq:three:17} \frac{F_{31}}{F_{21}} = \frac{n_1 C_{13}}{A_{31}} \frac{A_{21}}{n_1 C_{12}} \frac{E_{31}}{E_{21}} \frac{A_{31}}{A_{21}} \approx \frac{C_{13}}{C_{12}}
\end{equation}
So the flux rates depend \emph{only} on the collisional excitation rates in this limit. Re-expressing the flux ratio in terms of more fundamental quantities:
\begin{equation}
  \label{eq:three:18} \frac{F_{31}}{F_{21}} \approx \frac{C_{13}}{C_{12}} \approx \frac{\Omega_{13} e^{-E_{13}/kT}}{\Omega_{12} e^{-E_{12}/kT}} \approx \frac{\Omega_{13}}{\Omega_{12}} \approx \frac{g_{3}}{g_2}
\end{equation}
for somewhat unobvious reasons. So if we measure line strengths and get this magic number, all we really know is that we're in the low density limit. The magic comes in when we are in the high density limit:
\begin{equation}
  \label{eq:three:19} \frac{F_{31}}{F_{21}} = \frac{n_3}{n_2} \frac{A_{31}}{A_{21}} \frac{E_{31}}{E_{21}}
\end{equation}
Since in the high density limit there are many collisions, the number densities should be Boltzmann-like:
\begin{equation}
  \label{eq:three:20} \frac{n_3}{n_2} = \frac{g_3}{g_2} e^{-E_{23}/kT}\approx \frac{g_3}{g_2}
\end{equation}
So the flux ratio is now
\begin{equation}
  \label{eq:three:21} \frac{F_{31}}{F_{21}} \approx \frac{A_{31}}{A_{21}}\frac{g_3}{g_2}
\end{equation}
So far, this isn't really a density diagnostic, though we've identified the values at both limits. The transition between these limits should be at the critical density discusse earlier, defined by:
\begin{equation}
  \label{eq:three:22} A_{21} n_2 = n_e n_2 \alpha_{21}
\end{equation}
Earlier, we found that this critical density was found to be
\begin{equation}
  \label{eq:three:23} n_{e, \mathrm{crit}} = \frac{A_{21}g_2 T^{1/2}}{\beta \Omega_{12}}
\end{equation}
Different transitions will have different critical densities that depend on their intrinsic physics, for electron densities below the smaller critical density, the flux ratio is constant, and the same holds for electron densities greater than the larger critical density. In between, though, their is a transition zone wherein we can approximate a density. Examples of ions that allow for such a diagnostic are O$^+$, S$^+$, Ne$^{+3}$, and Ar$^{+3}$.
% subsection three_level_atom (end)
\subsection{Energy Level Diagrams} % (fold)
\subsubsection*{Suggested References} % (fold)
\label{ssub:suggested_references}
\begin{itemize}
  \item Draine Chapter 4
\end{itemize}
% subsubsection suggested_references (end)
\label{sub:energy_level_diagrams}
Often we will refer to ions by their atomic name followed by a roman numeral that is one greater than its oxidation state. For instance, O$^{+2}$ is referred to as O\,III.
% TODO Add energy level diagrams
There will be some energy level diagram figures after this demonstrating where some of these energy level assumptions work.\\

\n Let's briefly review the electronic configurations of some of these ions. O\,I has 8 electrons, so in its ground state it has an electronic configuration of $1s^2\ 2s^2\ 2p^4$. O\,III however, has a configuration of $1s^2\ 2s^2\ 2p^2$, where the outermost electrons are actually in different orbitals (remember Hund's rules!). This $p^2$ configuration disallows the two outermost electrons from having the same quantum numbers ($n,\ \ell,\ m,\ s$). In this case, we know that $n=2$ (the ``2'' in $2p^2$), and $\ell=\pm1$ (the $p$). For electrons, $s = \pm 1/2$. So the total spin can take on values of 0, or 1 and the total orbital angular momemtum can be 0, 1, or 2. We'll often see this configurations labeled via $\mathrm{^{2S+1}L_J}$ where S is the total spin angular momentum, L is the orbital name ($s,\ p,\ d,\ f,$ etc.), and J is the total angular momentum. The available states for O\,II are shown in Table~\ref{tab:oii}
\begin{table}
  \centering
  \begin{tabular}{l l l}
    $^1\mathrm{D}_{J=2}$ & L = 2 & S = 0\\
    $^3\mathrm{D}_{J=3,2,1}$ & L = 2 & S = 1 \\
    $^1\mathrm{P}_{J=1}$ & L = 1 & S = 0\\
    $^3\mathrm{P}_{J=2,1,0}$ & L = 1 & S = 1\\
    $^1\mathrm{S}_{J=0}$ & L = 0 & S = 0\\
    $^3\mathrm{S}_{J=1}$ & L = 0 & S = 1
  \end{tabular}
  \caption{Possible electron configurations for O\,II.}
  \label{tab:oii}
\end{table}
  Not all of these states are allowed transitions though, for reasons I still
  need to record here.
  % TODO Explain transition rules.

% subsection energy_level_diagrams (end)
\subsection{Infrared Line Diagnostics} % (fold)
\label{sub:infrared_line_diagnostics}
Transitions between fine-structure levels of $p^2$ and $p^4$ ions are the
dominant cooling processes in gas at $T = 100-3000$ K. Examples of such lines
are [CII] 158 $\mu$m and [OIII] 88.36 $\mu$m. Atmospheric water vapor blocks
these lines in the 25 to 300 $\mu$m range, but airborn observatories can
observe them, and ALMA is making great headway in observing mm-scale
radiation.\\
% TODO fill this out... missed information from the slides.

\n However, note that in the real world (like in \HII regions), the ISM is
clumpy, so these diagnostics actually only measure the number density of the
clumps, $n_{e,c}$. The emission measure, however, cares about the average
electron density: $\mathrm{EM}\sim \avg{n_e^2}\times(\mathrm{length})$. We
parameterize this clumpiness with a \textbf{volume filling factor} for the
clumps, $f$, defined by
\begin{equation}
  \label{eq:ff:1} \avg{n_e^2}\approx fn_{e,c}^2
\end{equation}
Typically we find $f\sim 0.1-0.1$ in nearby \HII regions.
% subsection infrared_line_diagnostics (end)
\subsection{Resonance Lines and Selection Rules} % (fold)
\label{sub:resonance_lines_and_selection_rules}
Remember that energy transitions are surrounded in square brackets when they are the so-called ``forbidden transitions''. The more run-of-the-mill resonance transitions, or allowed transitions, are those that obey the following rules:
\begin{enumerate}
  \item Only 1 electron involved in the transition
  \item Initial and final states have different parity
  \item Emitted photon carries 1 unit of angular momentum, so $\Delta L = \pm 1$
  \item Electron spin does not change
  \item Change in the total angular momentum of the active electron is $\Delta J = \pm 1$ or $0$, with $J=0\to0$ being forbidden.
\end{enumerate}
The statistical weight of any level is $g=2J + 1$.\\

\n In addition to the forbidden lines, which break rule 4, there are the intercombination, or \textbf{semi-forbidden} lines. These are transitions that break any of the other rules. These transitions are about a million times lower than the resonance transitions ($A\sim 10^2-10^3\ \mathrm{s^{-1}}$ compared to $\sim10^8\ \mathrm{s^{-1}}$ for resonance lines). Forbidden lines are several orders of magnitude less prevalent still, as they are magnetic dipole transitions.\\

\n For forbidden or semi-forbidden transitions to be important, we must have very low densities, since higher densities will make collisional processes dominate over these weak transitions. As such, the emitted photons from forbidden transitions are unlikely to be absorbed by another ion, so forbidden line photons usually escape from a nebula and are thus very important coolants. Examples of such transitions are [OIII] 4959, 5007 and [OII] 3726,29
\subsubsection{Hydrogen Line Series} % (fold)
\label{ssub:hydrogen_line_series}
The line transitions in a de-exciting electron (for instance, a recombination electron that came in at high $n$, $\ell$ to form an excited hydrogen atom) often produce a series of lines. Transitions that end at $n=1$ are called the \textbf{Lyman Series}. Transitions from $n=2$ to $n=1$ are Lyman-$\alpha$ transitions, and those from $n=3$ to $n=1$ are Lyan-$\beta$ transitions, etc. Similarly, transitions to the $n=2$ state form the Balmer series (denoted H$\alpha$, H$\beta4$, etc.), and the transitions to $n=3$ form the Paschen series.
% TODO transition from $n=2, \ell=0$ to $n=1, \ell=0$
% subsubsection hydrogen_line_series (end)
% subsection resonance_lines_and_selection_rules (end)
% section collisionally_excited_lines (end)
\section{Molecular Spectra} % (fold)
\label{sec:molecular_spectra}
\subsubsection*{Suggested Refeferences} % (fold)
\label{ssub:suggested_refeferences}
\begin{itemize}
  \item Draine Chapter 5
\end{itemize}
% subsubsection suggested_refeferences (end)
Molecules present additional complication since rotational and vibrational
energies become important. We'll first explore rotating molecules.
\subsection{Rotating Molecules} % (fold)
\label{sub:rotating_molecules}
Like nearly all forms of atomic energy, the energies of rotating molecules are
quantized to energy levels. These rotational energy levels are related tot he
moments of inertia of the molecules along the various axes of symmetry.

\paragraph{Eaample: Linear Diatomic Molecules} % (fold)
\label{par:eaample_linear_diatomic_molecules}
Suppose we have a linear diatomic molecule with characteristic lengthscale $r$
and reduced mass $\mu$. Then the moment of inertia of this molecule is
\begin{equation}
  \label{eq:ldmol:1} I = \mu r^2
\end{equation}
And the quantized rotational energy levels are
\begin{equation}
  \label{eq:ldmol:2} E = J(J+1)\frac{h^2}{8\pi^2 I}
\end{equation}
% TODO Finish this. This blackboard is in the worst position possible
% paragraph eaample_linear_diatomic_molecules (end)
% subsection rotating_molecules (end)
\subsection{Vibrating Molecules} % (fold)
\label{sub:vibrating_molecules}
% TODO Again, can't read for shit.
% subsection vibrating_molecules (end)
\subsection{Molecular Hydrogen} % (fold)
\label{sub:molecular_hydrogen}
H$_2$ does not radiate strongly because rotational transitions require a
heterogeneous linear molecule like CO. In the H$_2$ molecule, transitions occur
via electric quadrupole interaction. The least energetic transition is $J=0$ to
$J=2$. Thus, lifetiems of excited states are much, much longer than for the
ions (e.g.\ about 1000 years for the $J=2$ level). Hence, the rotational elvels
are populated by collisions.
% subsection molecular_hydrogen (end)
% section molecular_spectra (end)
\section{Transition Rates} % (fold)
\label{sec:transition_rates}
We return now to our two-level atom with two states, 1, and 2, separated by an
energy gap $E_{12}$. We can characterize the rate of decays from state 2
through some characteristic $e$-folding time via
\begin{equation}
  \label{eq:trans:1} N_2 = N_2(0)e^{-A_{21}t}
\end{equation}
Here, $A_{21}$ is proportional to the electric dipole matrix element, or
\begin{equation}
  \label{eq:trans:2} A_{21}\propto e^2\norm{\avg{\psi_1\vert\psi_2}}^2
\end{equation}
An atom with an electron in state 1 has a probabliity of absorbing a photon
given by
\begin{equation}
  \label{eq:trans:3} p_{12} = B_{12} u(\nu_{12})
\end{equation}
where $B_{12}$ is the Einstein absorption coefficient and $u(\nu)$ is the energy density in radiation at a given frequency. We usually derive this energy density from the \textbf{specific intensity}, $I_\nu$ ($I_\lambda$), which has units of energy per area per time per sold angle per frequency (wavelength). The specific intensity is a conserved quantity along a ray of light. To convert this intensity to a flux through some surface, we must integrate this quantity (really, the portion that is normal to the surface) over all incident angles, or
\begin{equation}
  \label{eq:trans:4} F_\nu = \int I_\nu\cos\theta\,d\Omega
\end{equation}
To get an energy density from the specific intensity (a process we'll discuss later), we use
\begin{equation}
  \label{eq:trans:5} u_\nu = \frac{4\pi}{c}I_\nu
\end{equation}
Similarly, the probability for stimulated emission to occur is
\begin{equation}
  \label{eq:trans:6} p_{21} = B_{21}u(\nu_{12})
\end{equation}
where the two Einstein coefficients are related via
\begin{equation}
  \label{eq:trans:7} g_1 B_{12} = g_2B_{21}
\end{equation}
Using LTE, we can derive the spontaneous emission Einstein coefficient to be
% TODO get from Draine Ch. 6

% section transition_rates (end)
%%%%%%%%%%%%%%%%%%%%%%%%%%%%%%%
% Wednesday, January 22, 2014 %
%%%%%%%%%%%%%%%%%%%%%%%%%%%%%%%
\section{Radiative Transfer} % (fold)
\label{sec:radiative_transfer}
\textit{Wednesday, January 22, 2014}
\subsection{Basic Definitions} % (fold)
\label{sub:basic_definitions}
\textbf{Energy flux} is the amount of radiative energy that passing through some surface per unit area per unit time. This is in contrast to the \textbf{specific intensity}, which is the amount of energy per unit time per unit area per unit solid angle per unit frequency (or sometimes per unit wavelength). Then the total energy through some bundle of rays is given by
\begin{equation}
  \label{eq:rt:1} dE = I_\nu\,dA\,dt\,d\Omega\,d\nu
\end{equation}
There are several quantities related to the specific intensity
\begin{enumerate}
  \item[(0)] \textbf{Mean Intensity}, defined by
  \begin{equation}
    \label{eq:rt:2} J_\nu \equiv \frac{1}{4\pi}\int I_\nu\,d\Omega
  \end{equation}
  which is just the specific unit averaged over all solid angles. Note that the differential of solid angle is
  \begin{equation}
    \label{eq:rt:3} d\Omega = \sin\theta\,d\theta\,d\phi
  \end{equation}
  This can be thought of as the zeroth moment of the specific intensity.
  \item[(1)] Correspondingly, the first moment of the specific intensity is the \textbf{specific flux}, given by
  \begin{equation}
    \label{eq:rt:4} F_\nu \equiv \int I_\nu \cos\theta\,d\Omega
  \end{equation}
  This is essentially the ``energy moment'' of the specific intensity.
  \item[(2)] While the first moment was an energy flux, the second moment is the momentum/pressure moment. Recalling that for a photon, $p = E/c$, we can get a momentum flux via
  \begin{equation}
    \label{eq:rt:5} P_\nu = \frac{1}{c}\int I_\nu \cos^2\theta\,d\Omega
  \end{equation}
  This is more familiarly known as the \textbf{radiation pressure}.
\end{enumerate}
As an example, consider a spaceship some distance $r$ from a (blackbody) star of radius $R$ and effective temperature $T$. We'd like to know what flux the spaceship observes from the star. Naively we may use the basic equation \eqref{eq:rt:4} to start:
% TODO Add diagram
\begin{align}
  \label{eq:rt:6} F_\nu &= \int B_\nu(T)\cos\theta\sin\theta\,d\theta\,d\phi\\
  \label{eq:rt:7} &= B_\nu(T) \int_0^{2\pi} d\phi\int_0^{\theta_{\mathrm{max}}} \cos\theta\sin\theta\,d\theta\\
  \label{eq:rt:8} &= \pi B_\nu(T)\left(\frac{R}{r}\right)^2
\end{align}
where $\theta_{\mathrm{max}}$ is half the angular diamter of the star as seen from the spaceship, $\theta_{\mathrm{max}} = \sin^{-1}(R/r)$.
% subsection basic_definitions (end)
\subsection{Equation of Radiative Transfer} % (fold)
\label{sub:equation_of_radiative_transfer}
% TODO Add diagram
Consider a source of radiation shining through some medium to an observer. The change in the specific intensity, in its most general, is given by
\begin{equation}
  \label{eq:rt:9} dI_\nu = -\kappa_\nu I_\nu\,ds + j_\nu\,ds
\end{equation}
Where $j_\nu$ is the \textbf{emissivity}, which tells us the rate of emission of radiation of that frequency per unit volume per unit solid angle, or
\begin{equation}
  \label{eq:rt:10} j_\nu \equiv \frac{1}{4\pi} n_u A_{u\ell}h\nu
\end{equation}
In addition to the emissivity, we must also consider the \textbf{attenuation}, $\kappa_\nu$, which measures the rate at which radiation is absorbed, which is obviously proportional to the incident radiation, $I_\nu$. The attenuation has an absorption term and an emission term:
\begin{align}
  \label{eq:rt:11} \kappa_\nu &= n_\ell\sigma_{\ell u}(\nu) - n_u\sigma_{u\ell}(\nu)\\
  \label{eq:rt:12} &= n_\ell \sigma_{\ell u}(\nu)\left[1 - \frac{n_u \sigma_{u \ell}}{n_\ell \sigma_{\ell u}}\right]\\
  \label{eq:rt:13} &= n_\ell \sigma_{\ell u}(\nu)\left[1 - \frac{n_u/n_\ell}{g_u/g_{\ell}}\right]
\end{align}
where in the last step we've used detailed balance to eliminate cross sections in favor of statistical weights. Now in LTE, the number densities of states are given by a Boltzmann distribution:
\begin{equation}
  \label{eq:rt:14} \frac{n_u}{n_\ell} = \frac{g_u}{g_\ell} e^{-h\nu/kT}
\end{equation}
So for sufficiently high frequencies ($h\nu \gg kT$), the second term in \eqref{eq:rt:13} can be neglected, giving us simply
\begin{equation}
  \label{eq:rt:15} \kappa_\nu \approx n_\ell \sigma_{\ell u}
\end{equation}
This approximation is excellent for the vast majority of astrophysical applications. A notable exception is masers, where stimulated emission is important (note that these are at low wavelengths; microwaves in fact).
\subsubsection{Example: Lyman Alpha Forest from AGNs} % (fold)
\label{ssub:example_lyman_alpha_forest_from_agns}
% TODO add diagram
Quasars at $z\approx 3$ emit strong lymnan alpha radiation that travels through a vast expanse of IGM. As it goes through, lyman alpha opacity robs the incident spectrum of lyman alpha radiation. The transition back to the ground state is a permitted transition, so it happens fast and emits photons isotropically. Thus, very few of the absorbed photons are re-emiited in the original direction. Assuming the emissivity vainishes along this path (no spontaneous production of Lyman alpha photons), we get
\begin{equation}
  \label{eq:rt:16} \frac{dI_\nu}{I_\nu} = -\kappa_\nu\,ds \qquad \Rightarrow \qquad \left. \ln I_\nu\right\vert_s^{\mathrm{obs}} = -\int_s^{\mathrm{obs}}\kappa_\nu\,ds
\end{equation}
which inspires us to define the \textbf{optical depth}, a dimensionless parameter characterizing the relative opaqueness of a path, via
\begin{equation}
  \label{eq:rt:17} \tau_\nu = \int ds\, \kappa_\nu
\end{equation}
this gives us another way to express \eqref{eq:rt:16}:
\begin{equation}
  \label{eq:rt:18} I_\nu = I_\nu(s) e^{-\tau_\nu(\mathrm{obs})}
\end{equation}
% subsubsection example_lyman_alpha_forest_from_agns (end)
\subsubsection{Optical Depth} % (fold)
\label{ssub:optical_depth}
Thus there are two limiting cases, the optically thin case, where $\tau \ll 1$ and the optically thick $\tau \gg 1$ case. A source viewed through an optically thin medium will suffer very little attenuation, whereas a source viewed through an optically thick medium will be almost totally attenuated.\\

\n With the optical depth defined, we can recast the equation of radiative transfer neatly in terms of it:
\begin{equation}
  \label{eq:rt:19} dI_\nu + I_\nu d\tau_\nu = \frac{j_\nu}{\kappa_\nu}d\tau_\nu
\end{equation}
to make things work out nicely, we multiply both sides of \eqref{eq:rt:19} by the integrating factor $\exp\left(\int_{\mathrm{obs}}^\ell\kappa_\nu\,ds\right)$ to get
\begin{equation}
  \label{eq:rt:20} \int_s^{\mathrm{obs}}d\left[e^{\int_{\mathrm{obs}}^\ell \kappa_\nu ds}I_\nu\right] = \int_0^{\tau_\nu} \frac{j_\nu}{\kappa_\nu} e^{\int_{\mathrm{obs}}^\ell \kappa_\nu ds}d\tau
\end{equation}
With some more jiggery-pokery, we get
\begin{equation}
  \label{eq:rt:21} \boxed{I_\nu(\mathrm{obs}) = I_\nu(s)e^{-\tau_\nu} + \int_0^{\tau_\nu} \frac{j_\nu}{\kappa_\nu} e^{-(\tau_\nu - \tau_\nu')}\ d\tau_\nu'}
\end{equation}
The first term just give the attenuated original specific intensity, and the second term adds in the sources along the path, each attenuated from its source position to the observer.\\

\n Now if the cloud in the medium is in LTE, where the photon emission rate is equal to the photon absorption rate, we have the emissivity of
\begin{equation}
  \label{eq:rt:22} j_\nu \equiv \kappa_\nu B_\nu
\end{equation}
If we are at a uniform tepmerature, $T$, the equation of radiative transfer becomes
\begin{equation}
  \label{eq:rt:23} I_\nu(\mathrm{obs}) = I_\nu(s)e^{-\tau_\nu} + B_\nu(T)\int_0^{\tau_\nu}e^{-(\tau_\nu - \tau_\nu')}d\tau'
\end{equation}
Now this is an integral we can do, so this simplifies to
\begin{equation}
  \label{eq:rt:24} \boxed{I_\nu(\mathrm{obs}) = I_\nu(s) e^{-\tau_\nu} + B_\nu(T) \left(1- e^{-\tau_\nu}\right)}
\end{equation}
If the very diffuse limit, $\tau_\nu \ll 1$, we may take first term expansions of the exponentials to get simply
\begin{equation}
  \label{eq:rt:25} I_\nu(\mathrm{obs}) = I_\nu(s)(1-\tau_\nu)
\end{equation}
And for the very thick cloud, where $\tau_\nu \gg 1$, the exponentials vanish, giving simply
\begin{equation}
  \label{eq:rt:26} I_\nu(\mathrm{obs}) = B_\nu(T)
\end{equation}
% subsubsection optical_depth (end)
\subsubsection{Radio Astronomy} % (fold)
\label{ssub:radio_astronomy}
For radio astronomy, we are typically in the regime where $h\nu\ll kT$, so stimulated emission is important. Radio astronomers often speak of a \textbf{brightness temperature}, which is the temperature a source would need to have the observed brightness at a particular frequency (this temperature need not be the same for each frequency). Inverting the Planck function, we can get an analytic expression for the brightness temperature:
\begin{equation}
  \label{eq: rt:26} T_B(\nu) = \frac{h\nu}{k} \ln\left[\frac{2h\nu^3}{c^2 I_\nu} + 1\right]{-1}
\end{equation}
Radio astronomers will also use an antennae temperature via
% TODO Explain this better
\begin{equation}
  \label{eq:rt:27} \boxed{T_A \equiv \frac{c^2}{2k\nu^2}I_\nu}
\end{equation}
The transfer equation becomes
\begin{equation}
  \label{eq:rt:28} \frac{c^2}{2\nu^2k}I_\nu = \frac{c^2}{2\nu^2 k}I_\nu(s) + \frac{B_\nu c^2}{2\nu^2k} \left(1-e^{-\tau_\nu}\right)
\end{equation}
or simply
\begin{equation}
  \label{eq:rt:29} T_A(\mathrm{obs}) = T_A(s) + T_c\left(1-e^{-\tau_\nu}\right)
\end{equation}
In the limit where the antennae temperature of the source is much greater than the cloud temperature and the cloud is optically thin, \eqref{eq:rt:29} simplifies to
\begin{equation}
  \label{eq:rt:30} T_A = T_{A,s}(1-\tau_\nu) + T_c\left(1-(1-\tau_\nu)\right) = T_{A,s} - (T_{A,s} - T_c)\tau_\nu
\end{equation}
In the optically thick limit, we just get
\begin{equation}
  \label{eq:rt:31} T_A(\mathrm{obs}) = T_c
\end{equation}
% subsubsection radio_astronomy (end)
% subsection equation_of_radiative_transfer (end)
\subsection{Resonance Line Transfer} % (fold)
\label{sub:resonance_line_transfer}
\textbf{MISSING: MONDAY, JANUARY 27, 2014 (EQUIVALENT WIDTH, CURVE OF GROWTH, ETC.)}\\

%%%%%%%%%%%%%%%%%%%%%%%%%%%%%%%
% WEDNESDAY, JANUARY 29, 2014 %
%%%%%%%%%%%%%%%%%%%%%%%%%%%%%%%
\n \textit{Wednesday, January 29, 2014}

\subsubsection{The Voigt Profile} % (fold)
\label{ssub:the_voigt_profile}
The equivalent width, in wavelength units (though not per unit wavelength), is given by
\begin{equation}
  \label{eq:ew:1} W_\lambda = \frac{\lambda^2}{c} \int (1-e^{-\tau_\nu}d\nu)
\end{equation}
where the optical depth is
\begin{equation}
  \label{eq:ew:2} \tau_nu = s\,N\,\phi(\nu)
\end{equation}
and $\phi(\nu)$ is the line profile.\\

\n From quantum mechanics, we remember that the width of a line is inversely proportional to the lifetime of the state via the uncertainty principle, $\Delta E \Delta t \sim h$. This natural shape is the Lorentzian, which is like a sharper Gaussian with stronger wings. In functional form, the natural line profile is then
\begin{equation}
  \label{eq:ew:3} \phi_\nu = \frac{4\gamma_{21}}{16\pi^2(\nu - \nu_{21})^2 + \gamma_{21}^2}
\end{equation}
For a line generated by a transition from state 2 to state 1. To get the width of this line, let's compute the half-width at half maximum (HWHM), or the frequency $\nu_{\mathrm{HM}}$ where $\phi(\nu_{\mathrm{HM}}) = 2/\gamma_{21}$:
\begin{equation}
  \label{eq:ew:4} \frac{4\gamma_{21}}{16\pi^2\Delta\nu^2+\gamma-{21}^2} = \frac{2}{\gamma_{21}} \qquad \Rightarrow \qquad \Delta \nu = \frac{\gamma_{21}}{4\pi}
\end{equation}
So the full width at half maximum (FWHM) is
\begin{equation}
  \label{eq:ew:5} \boxed{2\Delta \nu = \frac{\gamma_{21}}{2\pi}}
\end{equation}
For example, for the Hydrogen $2\to 1$ transition, we can look up $\gamma_{21} = 6.265\times 10^8\ \mathrm{s^{-1}}$, which corresponds to a line width or velocity of 
\begin{equation}
  \label{eq:ew:6} \Delta v = \frac{\Delta\nu}{\nu_{21}}c = 0.0121\ \mathrm{km\ s^{-1}} \left[\frac{\lambda_{21}\gamma_{21}}{7618\ \mathrm{cm\ s^{-1}}}\right]
\end{equation}
So this is a \emph{very} thin width, especially considering that velocities due to thermal motions are typically  much higher than this velocity dispersion. The true line profile then is given by a convolution of the particle velocity distribution and the natural Lorentizan distribution. In general, the velocity distribution of the particles is a Gaussian, and the convolution of a Gaussian and a Loretnzian distribution is called a \textbf{Voigt Profile}. Analytically, it looks like
\begin{equation}
  \label{eq:ew:7} \boxed{\phi_\nu^{\mathrm{Voigt}} = \frac{1}{\sqrt{2\pi}}\int \frac{dv}{\sigma_{v}}e^{-v^2/(2\sigma^2)}\frac{4\gamma_{21}}{16\pi^2\left[\nu-(1-v/c)\nu_{21}\right]^2 + \gamma_{21}^2}}
\end{equation}
Unfortunately, there is no analytic solution for this integral. Instead, we'll look at two regimes within the Voigt profile.
\paragraph{Near Line Center (Gaussian)} % (fold)
\label{par:near_line_center_gaussian_}
Near the line center, where the fraction in \eqref{eq:ew:7} is nearly constant, the profile appears to be a gaussian:
\begin{equation}
  \label{eq:ew:8} s = \sqrt{\pi} \frac{e^2}{m_e c} \frac{f_{12}\lambda_{21}}{b} e^{-v^2/b^2}
\end{equation}
where we've used the \textbf{Doppler Parameter} $b=\sqrt{2}\sigma$. 
% paragraph near_line_center_gaussian_ (end)

\paragraph{Wings} % (fold)
\label{par:wings}At large $\Delta \nu$, we get
\begin{equation}
  \label{eq:ew:9} s = \sqrt{\pi} \frac{e^2}{m_ec}\frac{f_{12}\lambda_{12}}{b} \left[\frac{1}{4\pi^{3/2}}\frac{\gamma_{21}\lambda_{21}}{b}\frac{b^2}{v^2}\right]
\end{equation}
% paragraph wings (end)

\paragraph{Transition} % (fold)
\label{par:transition}
The transition between these two regimes occurs around 
\begin{equation}
  \label{eq:ew:10} e^{z^2}=r\pi^{3/2} \frac{bz^2}{\gamma_{21}\lambda_{21}}
\end{equation}
where $z\equiv v/b$. For Ly\,$\alpha$, this corresponds to $z = 3.2$, or $\Delta v = 32\ \mathrm{km\ s^{-1}}(b/10\ \mathrm{km\ s^{-1}})$. Remember that for a Gaussian, the FWHM is $2\sqrt{2\ln\sigma}$\\
% paragraph transition (end)

\n Now if the absorbing medium is optically thick at a certain frequency (i.e., $\tau_nu > 1$, then define $x\equiv v/b$ and again $v = (\nu - \nu_{21})/\nu_{21}$). Define $x_1$ such that the optical depth is unity at that frequency. The optical depth dies off exponentially as we move away from the resonance frequency,
\begin{equation}
  \label{eq:ew:11}\tau_\nu = \tau_0 e^{-x^2}
\end{equation}
Solving for $x_1$ then gives us
\begin{equation}
  \label{eq:ew:12} x_1 = \sqrt{\ln \tau_0}
\end{equation}
Taking a Taylor expansion of the attenuation, we find
\begin{equation}
  \label{eq:ew:13} 1-e^{-\tau_\nu} \approx \begin{cases}
    1, &\text{ for }\abs{x} < x_1\\
    \tau_\nu, &\text{ for }\abs{x} \geq x_1
  \end{cases}
\end{equation}
Then the equivalent width in units of wavelength is
\begin{align}
  \label{eq:ew:14} \frac{W(\lambda)}{\lambda_{12}} &= 2\times \frac{\lambda_{12}}{c} \frac{\nu_{12}b}{c} \int_0^\infty (1-e^{-\tau_\nu})\,d\nu\\
  \label{eq:ew:15} &\approx \frac{2b}{c}\left[ \int_0^{x_1}dx + \int_0^\infty \tau_0e^{-x^2}dx - \int_0^{x_1}\tau_0e^{-x^2}dx\right]\\
  \label{eq:ew:16} &\approx \frac{2b}{c} \left[\sqrt{\ln\tau_0} + \tau_0\frac{\sqrt{\pi}}{2}\left\{1 - \mathrm{erf}\left(\sqrt{\ln\tau_0}\right)\right\}\right]
\end{align}
In thte optically thick limit, where $\tau_0\to\infty$, we may ignore the bracketed term, giving us simply
\begin{equation}
  \label{eq:ew:17} \boxed{\frac{W_\lambda}{\lambda} = \frac{2b}{c}\sqrt{\ln\tau_0}}
\end{equation}
where
\begin{equation}
  \label{eq:ew:18} \tau_0 = \frac{s N_1 \lambda_{12}}{\sqrt{\pi} b}
\end{equation}
Scaling to the Ly $\alpha$ line, 
\begin{equation}
  \label{eq:ew:19} \tau_\nu = 1 \qquad \Rightarrow \qquad \boxed{N = 8\times 10^{13}\ \mathrm{cm^{-2}}\frac{b}{10\ \mathrm{km\ s^{-1}}}\frac{1}{f} \frac{1216\,\mathrm{\AA}}{\lambda_{12}}}
\end{equation}
So the ``Lyman Alpha Forest'' clouds are all optically thin with column densities $N(\mathrm{H}\,I) < 10^{14}\ \mathrm{cm^{-2}}$. Then there are Lyman limit systems, which are optically thick. Then finally, there are the Damped Ly $\alpha$ Absorbers, or DLAs, which we explore now.\\

\n For \emph{very} optically thick clouds, where $\tau_0 > 10^4$, the attenuation goes as
\begin{equation}
  \label{eq:ew:20} 1-e^{-\tau} \approx \begin{cases}
    1, &\text{ if }\abs{x}\leq x_1\\
    \tau_0\frac{a}{\sqrt{\pi}}x^{-2}, &\text{ if }\abs{x}> x_1
  \end{cases}
\end{equation}
where
\begin{equation}
  \label{eq:ew:21} a = \frac{\gamma_{12}c}{b \nu_{12}}
\end{equation}
Then the equivalent width is
\begin{equation}
  \label{eq:ew:22} \boxed{\frac{W_\lambda}{\lambda} = \frac{2}{c} \sqrt{\gamma_{12}}{\lambda_{12}^2sN_1}}
\end{equation}
Again scaling to Ly $\alpha$, we get
\begin{equation}
  \label{eq:ew:23} W_\lambda = 7.3\ \mathrm{\AA}\ \sqrt{\frac{N(\mathrm{H\,I})}{10^{20}\ \mathrm{cm^{-2}}}}
\end{equation}
And now we've identified the three main modes in the curve of growth at small, intermediate, and high column depths.
% subsubsection the_voigt_profile (end)
\subsubsection{21 cm Line} % (fold)
\label{ssub:21_cm_line}
In the hydrogen atom, the electron-proton spin coupling causes a hyperfine splitting between the two spin states of the electron in neutral hydrogen. The photon emitted through a spin-flip transition has a wavelength of 21 cm. In temperature units, this corresponds to a 0.06816 K. Recalling that the CMB temperature is $T_{\mathrm{CMB}} = 2.7\ \mathrm{K}$, the Boltzmann distribution of the two spin states,
\begin{equation}
  \label{eq:21cm:1} \frac{n_2}{n_1} = \frac{g_2}{g_1} e^{-h\nu_{12}/kT}
\end{equation}
means that the distribution is largely determined by the statistical weights. Thus, $n_2/n_1 = 3$ pretty much always. Thus
\begin{equation}
  \label{eq:21cm:2} \frac{n_2}{n(\mathrm{H\,I})} = \frac{3}{4}; \qquad \frac{n_1}{n(\mathrm{H\,I})} = \frac{1}{4}
\end{equation}
The emissivity in the 21-cm line is
\begin{equation}
  \label{eq:21cm:3} \gamma_{21} = \frac{n_2 A_{21} h\nu_{21}}{4\pi} \phi(\nu) = \frac{3}{16\pi} n(\mathrm{H\,I})A_{21}h\nu_{21}\phi(\nu)
\end{equation}
And the absorption is
\begin{align}
  \label{eq:21cm:4} \kappa_\nu =& n_1\sigma_{12} - n_2\sigma_{21}\\
  \label{eq:21cm:5} &= n_1 \frac{g_2}{g_1} \frac{A_{21}}{8\pi} \lambda_{21}^2 \left[1 - e^{-h\nu_{21}/(kT)}\right]\phi(\nu)\\
  \label{eq:21cm:6} &\approx \frac{3}{32\pi} A_{21}\frac{hc \lambda_{12}}{kT} n(\mathrm{H\,I})\phi(\nu)
\end{align}
And the optical depth is
\begin{equation}
  \label{eq:21cm:7} \tau_\nu = \int_s^o\kappa_\nu\,ds
\end{equation}
For a Gaussian $\phi(\nu)$, we get
\begin{equation}
  \label{eq:21cm:8} \tau_\nu = 2.190 \frac{N(\mathrm{H\,I})}{10^{21}\ \mathrm{cm^{-2}}}\left[\frac{100\ \mathrm{K}}{T}\right]\left[\frac{\mathrm{km\ s^{-1}}}{b/\sqrt{2}}\right]e^{-v^2/b^2}
\end{equation}
% subsubsection 21_cm_line (end)
% subsection resonance_line_transfer (end)
% section radiative_transfer (end)
\end{document}

